\setupbackend[
export=yes,
xhtml=yes
]
\setupexport[
title={\type{SYNBOL}, \type{SYNEOL} tagging test},
subtitle={},
author={J.R.C.H.}%,
%cssfile=38-SYNBOL-SYNEOL-ln-tagging-export-templates.css%
]
\usemodule[amsl]
\usemodule[vim]
\writestatus{main}{we define Python}
\definefloat[Beispiel]
\setupfloat[Beispiel][width=.8\textwidth,location=middle,spacebefore=small,spaceafter=small]
\setupcaption[Beispiel][] 
\defineframedtext[PythonFramed][align=middle,frame=none,width=.9\textwidth]
\definevimtyping[Python][syntax=python,tab=4,numbering=yes,lines=split,option={packed,hyphenated},before={\startPythonFramed\switchtobodyfont[10pt]},after=\stopPythonFramed]
\traceexternalfilters
\writestatus{main}{we start now}
\starttext
%\exportparameter\c!cssfile
This is a little test whether modifcation in \type{\SYNBOL} and \type{\SYNEOL} have any sideeffects
when export is set to XML
\placeBeispiel[]
[code:complex_number_test]
{this is a more realistic example on how vimtyping with active linenumbers may be used}
{%
\startPython
a=12
def hello_world(hello)
    print("simon says:    {}".format(hello))
hallo_welt("how do youdo")
\stopPython%
}
\stoptext

