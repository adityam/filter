\usemodule[lilypond]

\setuplilypond[directory=output]

\starttext

\startlilypond
\layout{
  indent=0\mm
  ragged-right = ##f
}
\paper  {
  myStaffSize = #20
  #(define fonts
    (make-pango-font-tree "palatino"
    "palatino"
    "palatino"
               (/ myStaffSize 20))) 
  line-width=6\in
  oddFooterMarkup=##f
  oddHeaderMarkup=##f
  bookTitleMarkup = ##f
  scoreTitleMarkup = ##f
 }
global = {
  \key c \major
  \time 4/4
}

sopMusic = \relative c'' {
  c4 c c8[( b)] c4
}
sopWords = \lyricmode {
  hi hi hi hi
}

altoMusic = \relative c' {
  e4 f d e
}
altoWords =\lyricmode {
  ha ha ha ha
}

tenorMusic = \relative c' {
  g4 a f g
}
tenorWords = \lyricmode {
  hu hu hu hu
}

bassMusic = \relative c {
  c4 c g c
}
bassWords = \lyricmode {
  ho ho ho hø
}

\score {
  <<
    \new ChoirStaff <<
      \new Lyrics = "sopranos"
      \new Staff = "women" <<
        \new Voice = "sopranos" { \voiceOne << \global \sopMusic >> }
        \new Voice = "altos" { \voiceTwo << \global \altoMusic >> }
      >>
      \new Lyrics = "altos"
      \new Lyrics = "tenors"
      \new Staff = "men" <<
        \clef bass
        \new Voice = "tenors" { \voiceOne << \global \tenorMusic >> }
        \new Voice = "basses" { \voiceTwo << \global \bassMusic >> }
      >>
      \new Lyrics = "basses"
      \context Lyrics = "sopranos" \lyricsto "sopranos" \sopWords
      \context Lyrics = "altos" \lyricsto "altos" \altoWords
      \context Lyrics = "tenors" \lyricsto "tenors" \tenorWords
      \context Lyrics = "basses" \lyricsto "basses" \bassWords
    >>
    \new PianoStaff <<
      \new Staff <<
        \set Staff.printPartCombineTexts = ##f
        \partcombine
        << \global \sopMusic >>
        << \global \altoMusic >>
      >>
      \new Staff <<
        \clef bass
        \set Staff.printPartCombineTexts = ##f
        \partcombine
        << \global \tenorMusic >>
        << \global \bassMusic >>
      >>
    >>
  >>
}
\stoplilypond
\input zapf
\stoptext

