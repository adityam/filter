%D \module
%D   [     file=t-filter,
%D      version=2011.02.21,
%D        title=\CONTEXT\ User Module,
%D     subtitle=Filter,
%D       author=Aditya Mahajan,
%D         date=\currentdate,
%D    copyright=Aditya Mahajan,
%D        email=adityam <at> umich <dot> edu,
%D      license=Simplified BSD License]

\writestatus{loading}{ConTeXt User Module / Filter}

\startmodule    [filter]

\unprotect

%D \section {Initialization}
%D
%D \subsubject {Interface}
%D
%D The first step is to set the interface variables. This allows me to use 
%D \type{\c!filter} etc. in the module definition, and thereby reduces the risk
%D of a typo. Currently, only English names are provided. If someone wants, I
%D can also add other multi-lingual names.

\startinterface all
  \setinterfaceconstant {filter}           {filter}
  \setinterfaceconstant {filtercommand}    {filtercommand}
  \setinterfaceconstant {output}           {output} 
  \setinterfaceconstant {read}             {read} 
  \setinterfaceconstant {readcommand}      {readcommand} 
\stopinterface

\def\m!externalfilter{t-filter}

%D \subsubject {Messages}

\setinterfacemessage{externalfilter}{title}     {\m!externalfilter}
\setinterfacemessage{externalfilter}{notfound}  {file -- cannot be found}
\setinterfacemessage{externalfilter}{missing}   {output file missing}
\setinterfacemessage{externalfilter}{forbidden} {Fatal Error: Cannot use absolute path -- as directory}
\setinterfacemessage{externalfilter}{slash}     {Appending / to directory -- }

%D \subsubject {Name space}

\def\????externalfilter{@@@@externalfilter}
\def\!!!!externalfilter{externalfilter}
\def\externalfilter__temp_prefix{temp}
\def\currentexternalfilter{}
\def\externalfilter__count{\????externalfilter-\currentexternalfilter-counter}

\ifx\undefined\normalexpanded \let\normalexpanded\expanded \fi

\installparameterhandler     \????externalfilter \!!!!externalfilter
\installparameterhashhandler \????externalfilter \!!!!externalfilter
\installsetuphandler         \????externalfilter {externalfilters} %Note the plural

\doifmode\s!mkiv
  {\installattributehandler \????externalfilter \!!!!externalfilter}

%D \section {Tracing Macros}

\newif\iftraceexternalfilters

\let\traceexternalfilters\traceexternalfilterstrue

\starttexdefinition externalfilter__show_filenames
  \iftraceexternalfilters
    \writestatus\m!externalfilter{current filter : \currentexternalfilter}
    \writestatus\m!externalfilter{base file : \externalfilter__base_file}
    \writestatus\m!externalfilter{input file : \externalfilter__input_file}
    \writestatus\m!externalfilter{output file : \externalfilter__output_file}
  \fi
\stoptexdefinition

\def\externalfilter__show_status#1%
  {\iftraceexternalfilters
    \writestatus\m!externalfilter{#1}%
  \fi}

\starttexdefinition doshowfiltercommand
    \writestatus\m!externalfilter{command : \externalfilterparameter\c!filtercommand}
\stoptexdefinition

%D \section {The main user macros}

%D \subsubject {Define a new filter}

\def\defineexternalfilter
  {\dodoubleargument\externalfilter__define}

\def\externalfilter__define[#1][#2]%
  {\externalfilter__show_status{defining filter : #1}%
   \edef\currentexternalfilter{#1}%
   \getparameters[\????externalfilter#1][\s!parent=\????externalfilter,#2]%
   \doif{\externalfilterparameter\c!continue}\v!yes
      {\expandafter\newcounter\csname\externalfilter__count\endcsname}%
   \setvalue{\e!start#1}{\bgroup\obeylines\dodoubleargument\externalfilter__start[#1]}%
   \setvalue{\e!stop#1}{\externalfilter__process_filter}%
   \setvalue{process#1file}{\dodoubleargument\externalfilter__process_file[#1]}%
   \setvalue{inline#1}{\externalfilter__inline[#1]}
   }

\def\externalfilter__start[#1][#2]% filter options
  {% Initializations
   \egroup %\bgroup in \start#1 
   \edef\currentexternalfilter   {#1}%
   \begingroup % to keep assignments local
   \getparameters[\????externalfilter#1][\c!name=,#2]%
   \externalfilter__set_filenames
   % Capture the contents of the buffer
   \dostartbuffer[\externalfilter__temp_file][\e!start#1][\e!stop#1]}

\def\externalfilter__process_file[#1][#2]#3%
  {\begingroup
   \edef\currentexternalfilter    {#1}%
   \getparameters[\????externalfilter#1][\c!name=,#2]% 
   \externalfilter__set_directory
   \edef\externalfilter__input_file  {#3}%
   \splitfilename {#3}%
   %NOTE: \edef doesn not work
   \def\externalfilter__base_file   {\splitoffname}%
   % The output is always in the directory specified by 
   % \c!directory; even if the input is from some other directory
   \def\externalfilter__output_file{\externalfilter__get_directory\externalfilterparameter\c!output}%
   \externalfilter__show_filenames
   \externalfilter__execute_filter
   \externalfilter__read_processed_file
   \endgroup}
   
\def\externalfilter__inline[#1]%
  {\edef\currentexternalfilter {#1}%
   \begingroup % to keep assignments local
   \getparameters[\????externalfilter#1][\c!name=]% 
   \externalfilter__set_filenames
   \pushcatcodetable
   \futurelet\next\externalfilter__inline_aux}

%D \subsubject {Catcode tables}
%D 
%D Just to be sure, I define all catcode tables that are needed within the
%D module. Some of these are repetition of what is defined in ConTeXt, but the
%D internal names keep on changing which is a  maintenance nightmare.

\newcatcodetable \externalfilter__catcodes_read
\newcatcodetable \externalfilter__catcodes_write
\newcatcodetable \externalfilter__catcodes_verb

\startcatcodetable \externalfilter__catcodes_read % same as typcatcodesa
    \catcode\tabasciicode       \othercatcode
    \catcode\endoflineasciicode \othercatcode
    \catcode\formfeedasciicode  \othercatcode
    \catcode\spaceasciicode     \othercatcode
    \catcode\endoffileasciicode \othercatcode
    \catcode\leftbraceasciicode \begingroupcatcode
    \catcode\rightbraceasciicode\endgroupcatcode
\stopcatcodetable

\startcatcodetable \externalfilter__catcodes_write
  \catcode\backslashasciicode  = \escapecatcode
  \catcode\leftbraceasciicode  = \begingroupcatcode  
  \catcode\rightbraceasciicode = \endgroupcatcode
  \catcode\endoflineasciicode  = \activecatcode
  \catcode\formfeedasciicode   = \activecatcode
  \catcode\spaceasciicode      = \activecatcode
\stopcatcodetable 

\startcatcodetable \externalfilter__catcodes_verb % same as vrbcatcodes
    \catcode\tabasciicode      \othercatcode
    \catcode\endoflineasciicode\othercatcode
    \catcode\formfeedasciicode \othercatcode
    \catcode\spaceasciicode    \othercatcode
    \catcode\endoffileasciicode\othercatcode
\stopcatcodetable

%D \subsubject {Write argument to file verbatim}
%D
%D Surprisingly, there is nothing in the core to define a function that write its
%D argument to a file verbatim. I basically copied the \type{\type} macro.

\def\externalfilter__inline_aux
  {\ifx\next\bgroup
     \expandafter\externalfilter__inline_group
   \else
     \expandafter\externalfilter__inline_other
   \fi}

\def\externalfilter__inline_group
  {\setcatcodetable \externalfilter__catcodes_read
   \externalfilter__process_inline}

\def\externalfilter__inline_other#1%
  {\setcatcodetable \externalfilter__catcodes_verb
   \def\next##1#1{\externalfilter__process_inline{##1}}%
   \next}

\newwrite\externalfilter__write

\def\externalfilter__process_inline#1%
  {\immediate\openout \externalfilter__write\externalfilter__input_file
   \immediate\write   \externalfilter__write{\detokenize{#1}}%
   \immediate\closeout\externalfilter__write
   \popcatcodetable
   \externalfilter__execute_filter
   \endlinechar\minusone %to prevent line break after reading file
   \externalfilter__read_processed_file
   % Finalization
   \doif{\externalfilterparameter\c!continue}\v!yes
        {\doglobal\expandafter\increment\csname\externalfilter__count\endcsname}%
   \endgroup}



%D \section {Helper Functions}
%D
%D \subsubject {First and last character of a string}

\def\externalfilter__get_first_character#1%
  {\externalfilter__get_first_character_aux#1\relax}

\def\externalfilter__get_first_character_aux#1#2\relax{#1}

\def\externalfilter__get_last_character#1%
  {\@EA\externalfilter__get_last_character_aux#1\relax}

\def\externalfilter__get_last_character_aux#1#2%
  {\ifx#2\relax#1\else\@EA\externalfilter__get_last_character_aux\@EA#2\fi}

%D \subsubject {Set the name of output directory}

\def\externalfilter__set_directory
  {\edef\externalfilter__get_directory{\externalfilterparameter\c!directory}%
   \doifsomething{\externalfilter__get_directory}\externalfilter__set_directory_aux}
   
\def\externalfilter__set_directory_aux
  {\doif{\externalfilter__get_first_character\externalfilter__get_directory}{/}
      {\writeline
       \showmessage\!!!!externalfilter{forbidden}\externalfilter__get_directory
       \batchmode
       \errmessage{}
       \normalend}
   \doifnot{\externalfilter__get_last_character\externalfilter__get_directory}{/}
      {\showmessage\!!!!externalfilter{slash}\externalfilter__get_directory
       \edef\externalfilter__get_directory{\externalfilter__get_directory/}}}
  


%D \subsubject {Set file names}
%D
%D \type{\externalfilter__base_file} is the name of the temporary file without
%D extension. Its actual value depends on the state of \type{continue} key as
%D well as the value of \type{name} key.

\starttexdefinition externalfilter__set_filenames
   \externalfilter__set_directory
   % Set the name of temp file for the filter
   \doifelse{\externalfilterparameter\c!continue}\v!yes
       {\edef\externalfilter__temp_file{\externalfilter__temp_prefix-\currentexternalfilter-\csname\externalfilter__count\endcsname}}
       {\edef\externalfilter__temp_file{\externalfilter__temp_prefix-\currentexternalfilter}}
   \doifsomething{\externalfilterparameter\c!name}
       {\edef\externalfilter__temp_file{\externalfilter__temp_prefix-\currentexternalfilter-\externalfilterparameter\c!name}}
   % The following  macros are useful for filter= and filtercommand= options
   % The basename of the external file 
   \edef\externalfilter__base_file  {\jobname-\externalfilter__temp_file}%
   % In MkII, the buffer output is written to \TEXbufferfile{buffername} where
   % the macro \TEXbufferfile is defined as
   %
   %        \def\TEXbufferfile   #1{\bufferprefix#1.\f!temporaryextension}
   %
   % We redefine bufferprefix to include the directory name.
   \doifmode{\s!mkii}
      {\edef\bufferprefix{\externalfilter__get_directory\jobname-}}
   % In MkIV, we do not need to do such jugglary, because we can specify the
   % name of the file where the buffer has to be saved. This file is
   % \externalfilter__input_file (because it is the input to the filter).
   \edef\externalfilter__input_file {\externalfilter__get_directory\externalfilter__base_file.\f!temporaryextension}
   % The name of the file to which the filter output is written
   \edef\externalfilter__output_file{\externalfilter__get_directory\externalfilterparameter\c!output}
   \externalfilter__show_filenames
\stoptexdefinition


%D \subsubject {Process Filter}
%D
%D Execute filter, read the output and do book-keeping if needed.

\def\externalfilter__process_filter
  {% By defualt, buffers are in memory in MkIV
   \doifmode\s!mkiv{\savebuffer[\externalfilter__temp_file][\externalfilter__input_file]}%
   % Run external command
   \externalfilter__execute_filter
   \externalfilter__read_processed_file
   \endgroup 
   % Finalization
   \doif{\externalfilterparameter\c!continue}\v!yes
        {\doglobal\expandafter\increment\csname\externalfilter__count\endcsname}%
   \expanded{\checknextindentation[\externalfilterparameter\c!indentnext]}%
   \dorechecknextindentation
   }

%D \subsubject {Execute Filter}   

\def\externalfilter__execute_filter
  {\doshowfiltercommand
   \doifelse{\externalfilterparameter\c!continue}\v!yes
     {\doifmode{*first}
       {\executesystemcommand
        {mtxrun --ifchanged=\externalfilter__input_file\space 
            --direct \externalfilterparameter\c!filtercommand}}}
     {\executesystemcommand
        {\externalfilterparameter\c!filtercommand}}}

%D \subsubject {Read output}

\def\externalfilter__read_processed_file
  {\doif{\externalfilterparameter\c!read}\v!yes
      {\doiffileelse{\externalfilter__output_file}
          {\externalfilter__read_processed_file_aux}
          {\showmessage\!!!!externalfilter{notfound}\externalfilter__output_file 
           \blank
           {\tttf [[\getmessage\!!!!externalfilter{missing}]]}%
           \blank
          }}}

\def\externalfilter__read_processed_file_aux
  {\externalfilterparameter\c!before
   \begingroup
   \doifmode\s!mkiv
      {\dosetexternalfilterattributes\c!style\c!color}%
   \processcommacommand[\externalfilterparameter\c!setups]\directsetup
   \externalfilterparameter\c!readcommand\externalfilter__output_file
   \endgroup
   \externalfilterparameter\c!after}

%D \section {Default Values}

\setupexternalfilters
  [
   \c!before=,
   \c!after=,
   \c!style=,
   \c!color=,
   \c!indentnext=\v!auto,
   \c!setups=,
   \c!continue=\v!no,
   \c!read=\v!yes,
   \c!readcommand=\ReadFile,
   \c!directory=,
   \c!output=\externalfilterbasefile.tex,
   \c!filter=,
   \c!filtercommand={\externalfilterparameter\c!filter\space \externalfilter__input_file},
 ]

\def\externalfilterbasefile  {\externalfilter__base_file}
\def\externalfilterinputfile {\externalfilter__input_file}
\def\externalfilteroutputfile{\externalfilter__output_file}
   
\protect
\stopmodule
