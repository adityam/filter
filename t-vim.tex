\startmodule[vim]

\unprotect

\usemodule[filter]

\startinterface all
  \setinterfaceconstant {syntax}      {syntax} 
\stopinterface

\def\??vimsyntax??{vimsyntax}

\def\namedvimsyntaxparameter#1#2%
  {\executeifdefined{\??vimsyntax??::#1::#2}
     {\executeifdefined{\??vimsyntax??::#2}{}}}
  
\def\setupvimtyping
  {\dodoubleargument\dosetupvimtyping}

\def\dosetupvimtyping[#1][#2]%
  {\ifsecondargument
     \getparameters[\??vimsyntax??::#1::][#2]%
   \else
      \getparameters[\??vimsyntax??::][#1]%
   \fi}

\def\definevimtyping
  {\dodoubleargument\dodefinevimtyping}

\def\dodefinevimtyping[#1][#2]% 
  {\setupvimtyping[#1][#2]%
   \edef\currentvimtyping{#1}%
   \defineexternalfilter[#1]
    [\c!continue=yes,
     \c!read=\v!yes,
     \c!readcommand=\typefile,
     \c!output=\externalfilterbasefile.vimout,
     \c!filter={
      vim -u NONE % don't read global config file
          -e % run in ex mode
          -C % set compatible
          -n % no swap file
          -c "set tabstop=\namedvimsyntaxparameter\currentvimtyping\c!tab" %
          -c "syntax on" %
          -c "set syntax=\namedvimsyntaxparameter\currentvimtyping\c!syntax" %
          -c "let contextstartline=\namedvimsyntaxparameter\currentvimtyping\c!start" %
          -c "let  contextstopline=\namedvimsyntaxparameter\currentvimtyping\c!stop" %
          -c "source 2context.vim" % TODO: kpse:
          -c "wqa"},
    ]%
  }

\setupvimtyping
  [\c!tab=4,
   \c!start=1,
   \c!stop=0,
   \c!syntax=context,
  ]

\protect

\stopmodule

